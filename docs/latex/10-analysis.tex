\chapter{Аналитический раздел}


\section{Постановка задачи}
В соответствии с заданием на курсовую работу по курсу <<Операционные Системы>> необходимо разработать загружаемый модуль ядра Linux, предоставляющий информацию о процессах в системе за некоторый промежуток времени: их приоритетах, состояниях, времени выполнения, а также исполняющем ядре процессора. Для этого необходимо:

\begin{enumerate}
\item проанализировать работу планировщика Linux;
\item проанализировать структуры ядра, содержащие необходимую информацию;
\item проанализировать и выбрать методы передачи информации из модуля ядра в пространство пользователя;
\item разработать алгоритмы, используемые в программном обеспечении;
\item провести исследование с помощью разработанного программного обеспечения, планируются ли процессы проигрывания аудио- и видеофайлов, а также игровые и интерактивные процессы в ОС Linux как процессы реального времени..
\end{enumerate}

В результате загрузки разработанного модуля будет создан файл в файловой системе proc, содержащий эту информацию.


\section{Анализ работы планировщика}\label{descrsched}

Основной функцией планировщика является выбор задач на исполнение процессором. Каждая задача имеет свой алгоритм планирования и приоритет планирования sched\_priority. Планировщик принимает решения на основе алгоритма планирования и приоритета sched\_priority всех задач в системе \cite{bib:1}. Алгоритмы планирования можно разделить на 2 типа: обычные и алгоритмы реального времени.

Для задач, запланированных в соответствии с одним из обычных алгоритмов планирования, sched\_priority равен 0. Процессы, запланированные в соответствии с одним из алгоритмов реального времени, имеют значение sched\_priority в диапазоне от 1 (низкий) до 99 (высокий). 

Планировщик поддерживает список задач для каждого возможного значения sched\_priority. Чтобы определить, какую задачу выполнить следующей, планировщик ищет непустой список с наивысшим приоритетом sched\_priority и выбирает задачу из головы этого списка. Алгоритм планирования определяет, куда задача будет вставлена в список задач с равным приоритетом sched\_priority и как она будет перемещаться внутри этого списка. 

Планировщик поддерживает следующие 6 алгоритмов планирования \cite{bib:2}.
\begin{itemize}
    \item Обычные алгоритмы планирования. Для их реализации с версии 2.6.23 используется CFS (<<Completely Fair Scheduler>> -- <<Полностью честный планировщик>>), в основе которого упорядоченное красно-чёрное дерево, где все выполняемые задачи сортируются по ключу p->se.vruntime (взвешенное время использования процессора -- время использования процессора с учётом динамического приоритета). При планировании выбирается <<крайняя левая>> задача из этого дерева и выполняется на процессоре до тех пор, пока не найдётся задача в дереве, располагающаяся <<левее>>, чем текущая. 
    \begin{itemize}
        \item SCHED\_NORMAL (SCHED\_OTHER): используется для обычных задач;
        \item SCHED\_BATCH: алгоритм, при котором вытеснение происходит реже, чем при SCHED\_NORMAL, за счёт чего достигается более продолжительное выполнение одной задачи и лучшее использование кэша, однако при этом ухудшается интерактивность. 
        \item SCHED\_IDLE: может быть использован для низкоприоритетных задач.
    \end{itemize}
    \item Алгоритмы реального времени. 
    \begin{itemize}
        \item SCHED\_FIFO: задачи выбираются из соответствующей приоритету sched\_priority очереди и выполняются до тех пор, пока не будут заблокированы или вытеснены другой более приоритетной задачей; при переходе в состояние runnable или при изменении приоритета задача помещается в конец очереди.
        \item SCHED\_RR: усовершенствование SCHED\_FIFO, при котором каждой задаче разрешено выполняться только в течение некоторого временного интервала; после его истечения она будет помещена в конец списка своего приоритета.
        \item SCHED\_DEADLINE: доступна с версии 3.14; основана на алгоритме EDF; в основе алгоритма планирования лежит период P, соответствующий объявлению ядру о том, что Q единиц времени требуется для этой задачи каждые P единиц времени на любом ядре процессора.
    \end{itemize}
\end{itemize}


\section{Анализ структур ядра, предоставляющих информацию о процессах}

\subsection{Структура task\_struct}

Структура task\_struct в ядре Linux описывает каждый процесс в системе. \cite{bib:3}. Важные для данной работы поля структуры приведены в листинге \ref{lst:task_struct}. 

\begin{lstlisting}[label=lst:task_struct,caption=Структура task\_strcut.]
struct task_struct {
...
	unsigned int			__state;
	unsigned int			cpu;
	int				prio;
	int				static_prio;
	int				normal_prio;
	unsigned int			rt_priority;
	struct sched_entity		se;
    unsigned int			policy;
    struct sched_info		sched_info;
 	pid_t				pid;
	u64				utime;
	u64				stime;
 ...
 }
\end{lstlisting}

\paragraph*{Идентификатор процесса pid} Каждый процесс в операционной системе имеет свой уникальный идентификатор, по которому можно получить информацию об этом процессе, а также направить ему управляющий сигнал или завершить его.

\paragraph*{Состояние процесса \_\_state} каждый процесс в операционной системе в каждый момент времени находится в некотором состоянии. Поле \_\_state описывает это состояние. Соответствие значений этого поля состояниям показано в листинге \ref{lststate}. 
\begin{lstlisting}[label=lststate,caption=Соответствие значений поля \_\_state состояниям процесса]
#define TASK_RUNNING			0x00000000
#define TASK_INTERRUPTIBLE		0x00000001
#define TASK_UNINTERRUPTIBLE		0x00000002
#define __TASK_STOPPED			0x00000004
#define __TASK_TRACED			0x00000008
/* Used in tsk->exit_state: */
#define EXIT_DEAD			0x00000010
#define EXIT_ZOMBIE			0x00000020
#define EXIT_TRACE			(EXIT_ZOMBIE | EXIT_DEAD)
/* Used in tsk->state again: */
#define TASK_PARKED			0x00000040
#define TASK_DEAD			0x00000080
#define TASK_WAKEKILL			0x00000100
#define TASK_WAKING			0x00000200
#define TASK_NOLOAD			0x00000400
#define TASK_NEW			0x00000800
#define TASK_RTLOCK_WAIT		0x00001000
#define TASK_FREEZABLE			0x00002000
#define __TASK_FREEZABLE_UNSAFE	       (0x00004000 * IS_ENABLED(CONFIG_LOCKDEP))
#define TASK_FROZEN			0x00008000
#define TASK_STATE_MAX			0x00010000
\end{lstlisting}

\paragraph*{static\_prio} -- статический приоритет процесса ($[100; 139]$) не изменяется ядром при работе планировщика, однако может быть изменено пользователем с помощью макроса NICE\_TO\_PRIO. При создании процесса значение статического приоритета либо наследуется от родительского процесса, либо при наличии флага reset\_on\_fork выставляется по умолчанию, то есть становится равным $120$ \cite{bib:5}. 

% \begin{lstlisting}[label=lststaticprionice,caption=Определение статического приоритета процесса: макрос  NICE\_TO\_PRIO]
% include/linux/sched/prio.h

% #define MAX_NICE	19
% #define MIN_NICE	-20
% #define NICE_WIDTH	(MAX_NICE - MIN_NICE + 1)
% #define MAX_RT_PRIO		100
% #define MAX_PRIO		(MAX_RT_PRIO + NICE_WIDTH)
% #define DEFAULT_PRIO		(MAX_RT_PRIO + NICE_WIDTH / 2)
% #define NICE_TO_PRIO(nice)	((nice) + DEFAULT_PRIO)
% #define PRIO_TO_NICE(prio)	((prio) - DEFAULT_PRIO)
% \end{lstlisting}
% \newpage
% \begin{lstlisting}[label=lststaticprio,caption=Определение статического приоритета процесса: функция sched\_fork ]
% kernel/sched/core.c

% int sched_fork(unsigned long clone_flags, struct task_struct *p)
% {
% 	__sched_fork(clone_flags, p);
% 	p->__state = TASK_NEW;
% 	p->prio = current->normal_prio;
% 	uclamp_fork(p);
% 	/*
% 	 * Revert to default priority/policy on fork if requested.
% 	 */
% 	if (unlikely(p->sched_reset_on_fork)) {
% 		if (task_has_dl_policy(p) || task_has_rt_policy(p)) {
% 			p->policy = SCHED_NORMAL;
% 			p->static_prio = NICE_TO_PRIO(0);
% 			p->rt_priority = 0;
% 		} else if (PRIO_TO_NICE(p->static_prio) < 0)
% 			p->static_prio = NICE_TO_PRIO(0);

% 		p->prio = p->normal_prio = p->static_prio;
% 		set_load_weight(p, false);

% 		/*
% 		 * We don't need the reset flag anymore after the fork. It has
% 		 * fulfilled its duty:
% 		 */
% 		p->sched_reset_on_fork = 0;
% 	}

% ... }
% \end{lstlisting}

\paragraph*{rt\_priority} -- приоритет процесса реального времени ($[0; 99]$). При назначении алгоритма планирования rt\_priority приравнивается к sched\_priority (п. \ref{descrsched}). Значение этого приоритета определяет, является ли процесс задачей реального времени. 
% \newpage
% \begin{lstlisting}[label=lstrtprio,caption=Определение приоритета процесса реального времени]
% kernel/sched/core.c

% static void __setscheduler_params(struct task_struct *p,
% 		const struct sched_attr *attr)
% {
% 	int policy = attr->sched_policy;
% 	if (policy == SETPARAM_POLICY)
% 		policy = p->policy;
% 	p->policy = policy;
% 	if (dl_policy(policy))
% 		__setparam_dl(p, attr);
% 	else if (fair_policy(policy))
% 		p->static_prio = NICE_TO_PRIO(attr->sched_nice);
% 	p->rt_priority = attr->sched_priority; // ! setting of rt_priority
% 	p->normal_prio = normal_prio(p);
% 	set_load_weight(p, true);
% }
% \end{lstlisting}
\paragraph*{normal\_prio} -- нормальный приоритет ([-1; 139]) для обычных процессов равняется значению статического приоритета static\_prio; для процессов реального времени равняется значению, вычисленному с использованием максимального значения приоритета и приоритета процесса реального времени. Для обычных процессов значение в диапазоне $[100, 139]$, для процессов с алгоритмом SCHED\_FIFO и SCHED\_RR -- в диапазоне $[0; 99]$, для процессов с алгоритмом SCHED\_DEADLINE равно $-1$.

% \begin{lstlisting}[label=lstnormalprio,caption=Определение нормального приоритета процесса]
% kernel/sched/core.c
% static inline int __normal_prio(int policy, int rt_prio, int nice) {
% 	int prio;
% 	if (dl_policy(policy))
% 		prio = MAX_DL_PRIO - 1;
% 	else if (rt_policy(policy))
% 		prio = MAX_RT_PRIO - 1 - rt_prio;
% 	else
% 		prio = NICE_TO_PRIO(nice);
% 	return prio;
% }
% \end{lstlisting}

\paragraph*{prio} -- значение приоритета, которое использует планировщик. Для процессов реального времени равен rt\_priority (т.е. находится в диапазоне $[0;99]$), для обычных процессов равен normal\_prio (т.е. static\_prio, находится в диапазоне $[100;139]$. Чем ниже значение prio, тем выше приоритет процесса. Также на основании анализа значения этого приоритета функция rt\_task определяет принадлежность процесса к процессам реального времени. 

% \begin{lstlisting}[label=lstprio,caption=Функция определения приоритета процесса]
% kernel/sched/core.c

% static int effective_prio(struct task_struct *p)
% {
% 	p->normal_prio = normal_prio(p);
% 	/*
% 	 * If we are RT tasks or we were boosted to RT priority,
% 	 * keep the priority unchanged. Otherwise, update priority
% 	 * to the normal priority:
% 	 */
% 	if (!rt_prio(p->prio))
% 		return p->normal_prio;
% 	return p->prio;
% }
% \end{lstlisting}



% \begin{lstlisting}[label=lstisrt,caption=Функции определения принадлежности процесса к процессам реального времени с помощью prio]
% include/linux/sched/rt.h

% static inline int rt_prio(int prio)
% {
% 	if (unlikely(prio < MAX_RT_PRIO))
% 		return 1;
% 	return 0;
% }
% static inline int rt_task(struct task_struct *p)
% {
% 	return rt_prio(p->prio);
% }
% \end{lstlisting}

% Кроме того, существует функция, которая определяет принадлежность процесса к процессам реального времени с помощью анализа значения алгоритма планирования. Она представлена в листинге \ref{lstisrtpolicy}.

% \begin{lstlisting}[label=lstisrtpolicy,caption=Функции определения принадлежности процесса к процессам реального времени с помощью алгоритма планирования]
% include/linux/sched/rt.h

% static inline bool task_is_realtime(struct task_struct *tsk)
% {
% 	int policy = tsk->policy;
% 	if (policy == SCHED_FIFO || policy == SCHED_RR)
% 		return true;
% 	if (policy == SCHED_DEADLINE)
% 		return true;
% 	return false;
% }
% \end{lstlisting}

\paragraph*{policy} -- принимает значения от 0 до 6 (4 пока зарезервировано) и указывает на алгоритм, в соответствии с которым запланирована задача. Соответствие значений поля policy алгоритмам показано в листинге \ref{lstpolicy}.
\begin{lstlisting}[label=lstpolicy,caption=Соответствие значений поля policy алгоритмам планирования]
include/uapi/linux/sched.h

#define SCHED_NORMAL		0
#define SCHED_FIFO		1
#define SCHED_RR		2
#define SCHED_BATCH		3
/* SCHED_ISO: reserved but not implemented yet */
#define SCHED_IDLE		5
#define SCHED_DEADLINE		6
\end{lstlisting}

\paragraph*{utime} -- это время, проведенное в режиме пользователя и затраченное на запуск команд. Данное значение включает в себя только время, затраченное центральным процессором, и не включает в себя время, проведенное процессом в очереди на исполнение.

\paragraph*{stime} -- это время процессора, затраченное на выполнение системных вызовов при исполнении процесса.

\paragraph*{cpu} -- это номер процессора, на котором исполняется задача.

\subsection{Структура sched\_info}
sched\_info -- структура, которая предоставляет информацию о планировании процесса \cite{bib:3}. Данная структура представлена в листинге \ref{lst:si}.

\begin{lstlisting}[label=lst:si,caption=Структура sched\_info. ]
struct sched_info {
	/* Cumulative counters: */

	/* # of times we have run on this CPU: */
	unsigned long			pcount;

	/* Time spent waiting on a runqueue: */
	unsigned long long		run_delay;

	/* Timestamps: */

	/* When did we last run on a CPU? */
	unsigned long long		last_arrival;

	/* When were we last queued to run? */
	unsigned long long		last_queued;
};
\end{lstlisting}

\begin{itemize}
\item количество запусков процесса на исполнение центральным процессором (поле pcount);
\item количество времени, проведенного в ожидании на исполнение (поле run\_delay);
\item время последнего запуска процесса на исполнение центральным процессором (поле last\_arrival);
\item время последнего добавления процесса в очередь на исполнение (поле last\_queued).
\end{itemize}


\subsection{Структура sched\_entity}
Данная структура описывает процесс как единицу планирования \cite{bib:3}. Наиболее важными для данной работы являются поля, представленные в листинге \ref{lst:se}. 

\begin{lstlisting}[label=lst:se,caption=Структура sched\_entity. ]
struct sched_entity {
	/* For load-balancing: */
...
	u64				exec_start;
	u64				sum_exec_runtime;
	u64				vruntime;
...
};
\end{lstlisting}

Для обычных процессов поле vruntime является ключом в красно-чёрном дереве CFS (п. \ref{descrsched}) и определяется как взвешенная разница между временем последнего обновления статистики по времени текущей задачи (поле exec\_start) и временем в настоящий момент.

Поле sum\_exec\_runtime является суммарным временем выполнения на процессоре текущей задачи для всех алгоритмов планирования. 

% \begin{lstlisting}[caption=Обновление vruntime]
% kernel/sched/fair.c

% static void update_curr(struct cfs_rq *cfs_rq)
% {
% 	struct sched_entity *curr = cfs_rq->curr;
% 	u64 now = rq_clock_task(rq_of(cfs_rq));
% 	u64 delta_exec;
% ...
% 	delta_exec = now - curr->exec_start;
% 	if (unlikely((s64)delta_exec <= 0))
% 		return;
% 	curr->exec_start = now;
% ...
% 	curr->sum_exec_runtime += delta_exec;
% 	schedstat_add(cfs_rq->exec_clock, delta_exec);

% 	curr->vruntime += calc_delta_fair(delta_exec, curr);
% 	update_min_vruntime(cfs_rq);
% ...
% }
% \end{lstlisting}

% \begin{lstlisting}[caption=Обновление sum\_exec\_runtime]
% kernel/sched/sched.h

% static inline void update_current_exec_runtime(struct task_struct *curr,
% 						u64 now, u64 delta_exec)
% {
% 	curr->se.sum_exec_runtime += delta_exec;
% 	account_group_exec_runtime(curr, delta_exec);

% 	curr->se.exec_start = now;
% 	cgroup_account_cputime(curr, delta_exec);
% }
% \end{lstlisting}

% Данная функция вызывается в функциях обновления временной статистики для задач реального времени: 
% \begin{itemize}
%     \item при планировании в соотвествии с SCHED\_FIFO, SCHED\_RR  вызывается в update\_curr\_rt (файл kernel/sched/rt.c);
%     \item при при планировании в соотвествии с алгоритмом SCHED\_DEADLINE вызывается в update\_curr\_dl (файл kernel/sched/deadline.c).
% \end{itemize}

Важно отметить, что существует функция, которая корректирует значения utime и stime таким образом, чтобы сумма равнялась sum\_exec\_runtime. Она вызывается при обращении к /proc/pid/stat.

% \begin{lstlisting}[label=lst:proc_ops,caption=Структура proc\_ops.]
% /*
%  * Adjust tick based cputime random precision against scheduler runtime accounting. ick based cputime accounting depend on random scheduling timeslices of a task to be interrupted or not by the timer.  Depending on these circumstances, the number of these interrupts may be over or under-optimistic, matching the real user and system cputime with a variable precision. Fix this by scaling these tick based values against the total runtime accounted by the CFS scheduler.
%  * This code provides the following guarantees:
%  *   stime + utime == rtime
%  *   stime_i+1 >= stime_i, utime_i+1 >= utime_i
%  */
% void cputime_adjust(struct task_cputime *curr, struct prev_cputime *prev,
% 		    u64 *ut, u64 *st)
% {
% ...
% 	if (stime < prev->stime)
% 		stime = prev->stime;
% 	utime = rtime - stime;

% 	if (utime < prev->utime) {
% 		utime = prev->utime;
% 		stime = rtime - utime;
% 	}
%  ...
% }
% \end{lstlisting}

\section{Передача данных из пространства ядра в пространство пользователя}
Файловая система /proc представляет собой интерфейс ядра, который позволяет получать информацию о процессах и ресурсах, которые они используют. При этом используется стандартный интерфейс файловой системы и системных вызовов. Структура proc\_ops используется для определения обратных вызовов чтения и записи. Важные для данной работы поля структуры представлены в листинге \ref{lst:proc_ops} \cite{bib:3}.

\begin{lstlisting}[label=lst:proc_ops,caption=Структура proc\_ops.]
struct proc_ops {
...
	int	(*proc_open)(struct inode *, struct file *);
	ssize_t	(*proc_read)(struct file *, char __user *, size_t, loff_t *);
	int	(*proc_release)(struct inode *, struct file *);
 ...
};
\end{lstlisting}


Кроме того, из-за разных уровней привилегий пространства пользователя и пространства ядра для копирования блоков данных из пространства ядра в пространство пользователя необходимо использовать специальную функцию -- copy\_to\_user \cite{bib:3}, определённую в файле include/linux/uaccess.h.

\subsection*{Выводы}
В результате анализа кода ядра были определены структуры, содержащие необходимую информацию  о процессах в системе: task\_struct, sched\_info и sched\_entity. Кроме того, был определён механизм передачи данных из пространства ядра в пространство пользователя.

