\chapter*{ЗАКЛЮЧЕНИЕ}
\addcontentsline{toc}{chapter}{ЗАКЛЮЧЕНИЕ}

В соответствии с заданием на курсовой проект по курсу <<Операционные Системы>> был разработан загружаемый модуль ядра Linux, предоставляющий информацию о процессах в системе за некоторый промежуток времени: их приоритетах, состояниях, времени выполнения, а также исполняющем ядре процессора. Таким образом, цель была достигнута. Для её достижения были решены следующие задачи:

\begin{enumerate}
\item описана работа планировщика Linux;
\item проанализированы и выбраны структуры ядра, содержащие необходимую информацию;
\item проанализированы и выбраны методы передачи информации из модуля ядра в пространство пользователя;
\item разработаны алгоритмы, используемые в программном обеспечении;
\item проведено исследование с помощью разработанного программного обеспечения для выявленения того, планируются ли процессы проигрывания аудио- и видеофайлов, а также игровые и интерактивные процессы в ОС Linux как процессы реального времени.
\end{enumerate}

В результате проведенных исследований с помощью разработанного ПО было показано, что процессы проигрывания аудиофайлов, видеофайлов, а также игровые и интерактивные процессы в ОС Linux не планируются как задачи реального времени. Однако в ситуации инверсии приоритетов приоритет низкоприоритетного процесса повышается до приоритета высокоприоритетного процесса. 
